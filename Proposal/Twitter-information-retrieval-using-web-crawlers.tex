\documentclass[12pt]{article}

\usepackage{Twitter-information-retrieval-using-web-crawlers}


\usepackage{graphicx,url}

%\usepackage[brazil]{babel}   
\usepackage[latin1]{inputenc}  

     
\sloppy

\title{Twitter information retrieval using web crawlers}

\author{Luiz D. R. Fran\c{c}a\inst{1}}


\address{Departamento de Estat\'{i}stica e Inform\'{a}tica\\Universidade Federal Rural de Pernambuco
  (UFRPE)\\
  Recife -- PE -- Brazil
  \email{luizdaniel.r.f@gmail.com}
}

\begin{document} 

\maketitle

\begin{abstract} 
Twitter is a microblogging service that allows the users share ideas with short messages. The twites (the messages sent on Twitter are called twittes) are limited in length by 140 characters which makes the information extraction and retrieval difficult. This project will extend the RetriBlog framework \cite{Ferreira:13} to make this process easier. 
\end{abstract}

\section{Introduction}

Twitter is a social network that allows people to exchange 140-character messages. Nowadays many people use twitter to share their ideas and opinions, even companies use it to show their products and interact with their costumers. With the growth of social networks like Twitter, a huge amount of data is produced everyday, in a single second five hundreds million twitters are sent \cite{NetLiveStats}. This project is meant to develop a solution for retrieve information from Twitter using web crawlers.

\section{Problem}
As stated earlier, twittes are 140-character messages exchanged between hundreds of million users around the world. The small length of the twittes makes it hard to extract relevant information using traditional methods \cite{Sriram:10}.  With so much information uploaded everyday, it's very hard to get the most relevant twittes and extract information from it. The fact that the twittes  are only 140-characters long make it even harder. With messages full of abbreviations and slangs.

\section{Reason}
The information retrieved from Twitter has many applications like predict stock market \cite{Bollen:11} and monitor political sentiment \cite{Bermingham:11}.
This project is relevant because it will make easier to explore the richness of information on Twitter. Futhermore it's going to allow the retrieval of relevant information that otherwise would be very difficult to find due to the vastness of the contents uploaded everyday on Twitter. 

\section{Goal}
This project proposal is meant to develop a framework that can retrieve information from Twitter. We are going to extend the RetriBlog framework \cite{Ferreira:13}. that is a framework to extract.



\section{Related Jobs}
These are some works related to information retrieval and extraction of information:
\begin{itemize}
\item Short Text Classification in Twitter to Improve Information Filtering \cite{Sriram:10}
\item Managing the Acronym/Expansion Identification Process for Text-Mining Applications \cite{Roche:08} 
\item Part-of-Speech Tagging for Twitter: Annotation, Features, and Experiments \cite{Gimpel:11}
\end{itemize}

\bibliographystyle{sbc}
\bibliography{sbc-template}

\end{document}
